\documentclass[12pt, c]{beamer}
% Setup appearance:

% Standard packages
\usepackage[francais]{babel}
\usepackage[latin1]{inputenc}
\usepackage{times}
\usepackage[T1]{fontenc}
\usepackage{float}
\usepackage{multirow}
\usepackage{subfigure}
\usepackage{pifont}
\usepackage{tikz}
\usepackage{morewrites}
%\usepackage{subfig}
\usetikzlibrary{arrows}
\tikzstyle{block}=[draw opacity=0.7,line width=1.4cm]
\DeclareOption{english}{\trans@use@and@alias{english}{English}}
\ProcessOptions*

\mode<presentation> {
	\definecolor{bgcol}{RGB}{255,255,255}
	\definecolor{grencol}{RGB}{0,102,0}
	\definecolor{rulecol}{RGB}{3,16,67}
	\definecolor{boxtitre}{RGB}{37,45,137}
	}

  \usetheme{Warsaw}
	%\usecolortheme[named=rulecol]{structure}
	\usefonttheme[onlysmall]{structurebold}
	\usefonttheme{structureitalicserif}
	\useinnertheme{rounded}
	\useoutertheme{split}
	
	\beamertemplateshadingbackground{bgcol}{bgcol} % pour jouer sur la couleur % du fond
	\beamertemplatetransparentcovereddynamic
	\setbeamertemplate{background canvas}[vertical shading][top=white, bottom=white!60!white]
	%\setbeamercolor{normal text}{bg=rulecol,fg=black}
	%\setbeamercolor{title in head}{bg=rulecol!80,fg=bgcol}
	\setbeamercolor{title in foot}{bg=black,fg=white}
	\setbeamercolor{author in head}{bg=rulecol,fg=bgcol}
	\setbeamercolor{author in foot}{bg=black!80,fg=white}
	\setbeamercolor{section in head}{bg=black,fg=bgcol}
	\setbeamercolor{section in foot}{bg=rulecol!80,fg= bgcol}
	\setbeamercolor{subsection in head}{bg=black,fg=bgcol}
	\setbeamercolor{subsection in foot}{bg=rulecol,fg=bgcol}
	\setbeamercolor{logo}{bg=rulecol,fg=white}
	\setbeamercolor{section in head/foot}{bg=black!90,fg=white}
	\setbeamercolor{subsection in head/foot}{bg=rulecol!90,fg=white}
	\setbeamercolor{boiterouge}{bg=red!60,fg=black}
	\setbeamercolor{boiterougeB}{bg=red,fg=black}
	\setbeamercolor{boiteblanche}{bg=white,fg=black}
	\setbeamercolor{boiteverte}{bg=green,fg=black}
	\setbeamercolor{boiteblue}{bg=rulecol,fg=white}
	
	\setbeamerfont*{frametitle}{size=\normalsize}
	
	
  \setbeamertemplate{navigation symbols}{}
	\setbeamertemplate{itemize item}[square]
	\setbeamertemplate{enumerate item}[ball]
	\setbeamertemplate{itemize subitem}[triangle]
	\setbeamertemplate{itemize subsubitem}[circle]
	\logo{\includegraphics[height=5mm]{images/logouniv.png}}
	\setbeamertemplate{footline}{
		\leavevmode%
		\hbox{\hspace*{-0.06cm}
		
		\begin{beamercolorbox}[wd=.265\paperwidth,ht=2.25ex,dp=1ex,left]{author in foot}%
			\usebeamerfont{author in head/foot}\insertshortauthor%~~(\insertshortinstitute)
		\end{beamercolorbox}%
		
		\begin{beamercolorbox}[wd=.6\paperwidth,ht=2.25ex,dp=1ex,center]{title in foot}%
			\usebeamerfont{title in head/foot}\insertshorttitle
		\end{beamercolorbox}%
		
		\begin{beamercolorbox}[wd=.135\paperwidth,ht=2.25ex,dp=1ex,right]{section in foot}%
			\usebeamerfont{logo}\textcolor[rgb]{1,0.41,0.13}{\insertshortdate{}}\hspace*{0.4em}
			\insertframenumber{} / \inserttotalframenumber
		\end{beamercolorbox}}%
		
		\vskip0pt%
	}
\renewcommand{\raggedright}{\leftskip=0pt \rightskip=0pt plus 0cm}

\title[Secure Distributed Cluster Formation in Wireless Sensor Networks]{}

\author[$\;\;$ TCHIO AMOUGOU Styves Daudet]{}

\date[Janvier 2021]{janvier 2021}

%\subject{M�moire de Master 2}
\begin{document}
\begin{frame}
\transfade
	\vspace{-0.3cm}
	\begin{figure}
		\begin{center}
		\includegraphics[width=11cm,height=1.2cm]{images/enteteUds.PNG}
		\end{center}
	\end{figure}
	
	\begin{center}
	\tiny{\textsc{\textcolor{black}{DSCHANG SCHOOL OF SCIENCES AND TECHNOLOGY}}}\\
		\tiny{Unit� de Recherche en Informatique Fondamentale, Ing�nierie et  Application (URIFIA)}
	\end{center}
	\vspace{0.05cm}
\fcolorbox{boxtitre}{boxtitre}{\parbox{1\linewidth}{

\vspace{-0.3cm}
\begin{center}

\vspace{0.05cm}
 \textcolor{bgcol}{Secure Distributed Cluster Formation in Wireless Sensor Networks}

%\vspace{0.1cm}
\end{center}

\vspace{-0.1cm}
}}
\begin{center}
\fcolorbox{bgcol}{bgcol}{
\parbox{1\linewidth}{
\begin{center}
\vspace{-0.5cm}
\footnotesize Pr�sent� par :\\ \textbf{\textcolor{boxtitre}{ TCHIO AMOUGOU Styves daudet}} \\
\scriptsize{\textit{Matricule : CM-UDS-14SCI0251}}\\
 \tiny{\textit{Licenci� en Informatique Fondamentale}}\\
 \vspace{0.3cm}
				\scriptsize{\textit{Sous la direction de }}\\
				\scriptsize{\textbf{Dr BOMGNI ALAIN Bertand} }\\
								\tiny{(\textit{Charg� de Cours, Universit� de Dschang})}\\
				\end{center}
}}
\end{center}
\end{frame}

\begin{frame}{\scriptsize Sommaire}
\transwipe
\scriptsize
  \tableofcontents
\end{frame}

\section{Introduction}
	\subsection{Contexte}
	\setbeamercovered{invisible}
	\begin{frame}{Context}
	\transwipe
	
	\vspace{-0.25cm}
	\begin{block}
		\scriptsize Dans les r�seaux de capteur les attaques malicieux sont un probl�me r�el Et plusieurs protocoles propos�s ne r�sistent pas au attaque malicieuse dans des environnement hostile
	\end{block}
	\end{frame}
			

			
	\subsection{Probl�matique g�n�rale}		
			\begin{frame}{Probl�matique g�n�rale}
	\transglitter
				\begin{alertblock}{ Probl�matique g�n�rale}
					\large d�tection des noeuds  malicieux dans les r�seaux de capteur.
				\end{alertblock}
				\begin{figure}%
				\centering
				\includegraphics[width=2cm, height=3cm]{images/bonhomme_question.JPG}%
				\end{figure}
			\end{frame}	
						
\section{ secure distributed cluster formation in wireless sensor networks}	
			\begin{frame}{}
				\transfade
				\begin{center}
					\vspace{-0.2cm}
				\huge\textbf{ \textsc{ secure distributed cluster formation in wireless sensor networks}}
				\end{center}
			\end{frame}
			\subsection{Propri�t�s}
			\begin{frame}{propri�t�s}
				\transwipe
				\vspace{-0.21cm}
					\begin{block}{\scriptsize Our secure distributed cluster formation protocol has the following properties even if there are external and insider attackers}
						\begin{itemize}
							\item[\ding{43}]The protocol is fully distributed. Each node computes its clique only using the information from its 1-hop neighbors;
							\item[\ding{43}] The protocol is guaranteed to terminate. Participating nodes that do not follow the protocol specification (e.g., send conflicting messages) will be identified and removed from all cliques; 
						\item[\ding{43}] After the protocol terminates, all normal nodes are divided into mutually disjoint cliques. All normal nodes are guaranteed to have consistent views on their clique memberships even in hostile environment; 
						\end{itemize}
					\end{block}
					\centering
						\includegraphics[width=2cm, height=3cm]{images/bonhomme_idea.JPG}
			\end{frame}
			\begin{frame}{propri�t�s}
				\transwipe
					\vspace{-0.21cm}
						\begin{block}{\scriptsize Our secure distributed cluster formation protocol has the following properties even if there are external and insider attackers}
							\begin{itemize}
								\item[\ding{43}]normal nodes are divided into mutually disjoint cliques;
								\item[\ding{43}] all the normal nodes in each clique agree on the same clique memberships;
								\item[\ding{43}]while external attackers can be prevented from participating in the cluster formation process, inside attackers that do not follow the protocol semantics can be identified and removed from the network;
								\item[\ding{43}]the communication overhead is moderate;
							\end{itemize}
						\end{block}
						\centering
						\includegraphics[width=2cm, height=3cm]{images/bonhomme_idea.JPG}
			\end{frame}
			\subsection{Assertion}
				\begin{frame}{Assertion}
					\transwipe
					\vspace{-0.21cm}
					\begin{block}{}
							\begin{itemize}
								\item[\ding{43}] We assume each node knows its 1-hop neighbors
								\item[\ding{43}] We assume the sensor nodes can perform public key based digital signature operations.

								\item[\ding{43}] We assume the clocks of the normal nodes are loosely synchronized, as required by TESLA.

								\item[\ding{43}]We also assume the public keys used by the sensor nodes are properly authenticated
 
							\end{itemize}
					
					\end{block}
				\end{frame}
			\subsection{Protocol Specification}
				\begin{frame}{Protocol Specification}
					\transwipe
					\vspace{-0.21cm}
					\begin{block}{The protocol is summarized below:}
							\begin{itemize}
								\item[\ding{43}] Step 1: Each node exchanges its neighbor lists with its neighbors, and computes its local maximum clique.
								\item[\ding{43}] Step 2: Each node exchanges its local maximum clique with its neighbors, and updates its maximum clique according to its neighbor nodes? local maximum cliques.

								\item[\ding{43}] Step 3: Each node exchanges the updated clique with its neighbors, and derives its final clique.

							\end{itemize}
					\end{block}
				\end{frame}
				\begin{frame}{Protocol Specification}
					\transwipe
					\vspace{-0.21cm}
					\begin{block}{The protocol is summarized below:}
							\begin{itemize}
								\item[\ding{43}]Step 4: Each node exchanges the final clique with its neighbors.
								
									\begin{itemize}
										\item If no clique inconsistency is detected, it terminates successfully.
										\item Otherwise, it enters Step 5.
									\end{itemize}
								
								 
								\item[\ding{43}]Step 5: Each node performs conformity checking. 
									\begin{itemize}
										\item If it identifies malicious (neighbor) nodes, it removes them from the network, and restarts the protocol from Step 1.
										\item Otherwise, it enforces the clique agreement and terminates.
									\end{itemize}									 

							\end{itemize}
					\end{block}
				\end{frame}
				\subsection{Limite}
					\begin{frame}{Limite}
						\transwipe
						\vspace{-0.21cm}
						\begin{block}{}
							\begin{itemize}
								\item[\ding{43}] Currently, our protocol is suitable for static sensor networks, in which nodes do not move frequently
							\end{itemize}
						\end{block}
					\end{frame}
			
	
		
		\begin{frame}
\transdissolve
	\vspace{1cm}
	\begin{center}
	\huge \textbf{\textsc{Merci pour votre aimable attention}}
	\end{center}
\end{frame}
	
%\bibliographystyle{apacite} 
%\bibliography{References/references}

%\subject{M�moire de Master 2}
\begin{frame}
\transfade
	\vspace{-0.3cm}
	\begin{figure}
		\begin{center}
		\includegraphics[width=11cm,height=1.2cm]{images/enteteUds.PNG}
		\end{center}
	\end{figure}
	
	\begin{center}
	\tiny{\textsc{\textcolor{black}{DSCHANG SCHOOL OF SCIENCES AND TECHNOLOGY}}}\\
		\tiny{Unit� de Recherche en Informatique Fondamentale, Ing�nierie et  Application (URIFIA)}
	\end{center}
	\vspace{0.05cm}
\fcolorbox{boxtitre}{boxtitre}{\parbox{1\linewidth}{

\vspace{-0.3cm}
\begin{center}

\vspace{0.05cm}
 \textcolor{bgcol}{Secure Distributed Cluster Formation in Wireless Sensor Networks}

%\vspace{0.1cm}
\end{center}

\vspace{-0.1cm}
}}
\begin{center}
\fcolorbox{bgcol}{bgcol}{
\parbox{1\linewidth}{
\begin{center}
\vspace{-0.5cm}
\footnotesize Pr�sent� par :\\ \textbf{\textcolor{boxtitre}{ TCHIO AMOUGOU Styves daudet}} \\
\scriptsize{\textit{Matricule : CM-UDS-14SCI0251}}\\
 \tiny{\textit{Licenci� en Informatique Fondamentale}}\\
 \vspace{0.3cm}
				\scriptsize{\textit{Sous la direction de }}\\
				\scriptsize{\textbf{Dr BOMGNI ALAIN Bertand} }\\
								\tiny{(\textit{Charg� de Cours, Universit� de Dschang})}\\
				\end{center}
}}
\end{center}
\end{frame}
\end{document}